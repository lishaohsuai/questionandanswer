\chapter{时间复杂度的计算}
\label{chap11}
\begin{itemize}[noitemsep,topsep=0pt,parsep=0pt,partopsep=0pt]
	\item 知识点:讲解相关知识点。
	\item 题型:直接上真题。
\end{itemize}

\section{知识点和方法论}

\subsection{知识点}
\begin{itemize}[noitemsep,topsep=0pt,parsep=0pt,partopsep=0pt]
	\item 时间复杂度常用大 O 符号表示
	\item 时间复杂度是去掉最高项多项式前面的系数,且不包括函数的低阶项。
	\item 常见时间复杂度$Ο(1)<Ο(log_2 n)<Ο(n)<Ο(nlog_2 n)<Ο(n^2)<Ο(n^3)<Ο(2^n)<O(n!)<O(n^n)$
	\item 空间复杂度如果,算法原地工作那么,空间复杂度为O(1)
\end{itemize}

\subsection{方法论}
\begin{itemize}[noitemsep,topsep=0pt,parsep=0pt,partopsep=0pt]
	\item 简单计算相关的要计算的函数关键语句执行的次数。
	\item 对于关键语句使用令 关键语句执行的次数是 t 次。
\end{itemize}

\section{真题实战}

\subsection{2011年408}
设n是描述问题规模的非负整数,下面程序片段的时间复杂度是( )
\begin{lstlisting}[basicstyle=\small\ttfamily, caption={}, numbers=none]
x = 2;
while(x < n/2)
	x=2*x; // 3
\end{lstlisting}
解:\newline
{\color{red}(计算函数关键语句的执行次数)}\newline
令: 第三行语句执行了t次,\newline
可知条件不满足的情况是$ x*2^t >= n/2 $,其中x = 2\newline
所以条件不满足的情况 $ 2^{t+1} >= n/2  $ \newline
求解t: 可知 $t >= log_2(n) - 2$ \newline
舍去低阶项,可知 $ t >= log_2(n)$ \newline
再用大O表示,可得时间复杂度为 $O(log_2 n)$

\subsection{2014年408}
下面程序片段的时间复杂度是( )
\begin{lstlisting}[basicstyle=\small\ttfamily, caption={}, numbers=none]
count=0;
for(k=1; k<=n; k*=2)// 1
	for(j=1; j<=n; j++) // 2
		count++;// 3
\end{lstlisting}
解:\newline
{\color{red}(计算函数关键语句的执行次数)}\newline
计算 3 语句的频度\newline
已知:for里面for循环导致3语句执行次数是两个for次数相乘\newline
令 1语句中 k*=2 执行了 t 次。
1 语句可以执行 $ k*2^t > n $ \newline
可得 $ t > log_2 (n) $ \newline
1的每个循环中,易知 2 语句中 j++ 执行了 n 次。\newline
易知,3语句的执行次数和2语句中 j++是一样的的。\newline
得到总执行次数$ log_2 (n) * n$ (前面的系数忽略)\newline
易知,时间复杂度为 $ log_2 (n) * n $\newline

\subsection{2017年408}
下列函数的时间复杂度是( )
\begin{lstlisting}[basicstyle=\small\ttfamily, caption={}, numbers=none]
int func(int n){
	int i=0; sum=0;
	while(sum < n) 
		sum += ++i;\\ 4
	return i;
}
\end{lstlisting}
解:\newline
{\color{red}(计算函数关键语句的执行次数)}\newline
可知i的变化是 1,2,3,4,5...\newline
令 4 语句执行了 t 次\newline
sum = 1 + 2 + 3 + ... + t \newline
可知当条件不满足时$ sum = \frac{t * (1 + t)}{2} >= n $\newline
得知 $ t + t^2 >= 2n$\newline
忽略低次项$ t >= \sqrt{2*n} $\newline
忽略常数项的系数 $ t >= \sqrt{n}$ \newline
得知时间复杂度是 $ O(\sqrt{n})$






