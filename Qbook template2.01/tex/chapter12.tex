\chapter{线性表队列栈的计算}
\label{chap12}
\begin{itemize}[noitemsep,topsep=0pt,parsep=0pt,partopsep=0pt]
	\item 知识点:讲解相关知识点。
	\item 题型:直接上真题。
\end{itemize}

\section{知识点和方法论}

\subsection{知识点}
\begin{itemize}[noitemsep,topsep=0pt,parsep=0pt,partopsep=0pt]
	\item 线性表中的位子序列从{\color{red}1}开始,计算机中的数组从{\color{red}0}开始。
	\item 卡特兰数:n个不同元素入栈,出栈序列的个数为 $\frac{1}{n+1} C^{n}_{2n} = \frac{1}{n+1}\frac{(2n)!}{n! \times n!} $
	\item 后缀表达式也成为,逆波兰式。
	\item 在考试中简单的栈实现和队列实现
	\begin{itemize}[noitemsep,topsep=0pt,parsep=0pt,partopsep=0pt]
		\item 申明一个栈并初始化 int stack[maxSize]; int top=-1;
		\item 元素入栈 stack[++top] = x;
		\item 元素出栈 x= stack[top--];
	\end{itemize}
\end{itemize}

\subsection{方法论}
\begin{itemize}[noitemsep,topsep=0pt,parsep=0pt,partopsep=0pt]
	\item 如何想不出来只遍历一遍的方法,使用普通方法也可得到大部分的分数。
	\item 中缀表达式转为后缀表达式:\newline
		1. 栈中只存放符号\newline
		2. 处理括号\newline
			i. 若为 '(' 存入栈中。\newline
			ii. 若为 ')' 将 '(' 之前的符号依次弹出\newline
		3. 运用四则运算法则,比如\newline
			i.  栈顶'+' 小于扫描到的 '/' , '/' 入栈\newline
			ii. 栈顶'/' 大于扫描到的 '+' , '/' 出栈, '+' 入栈\newline
		4. 检验从后往前依次读取计算,看是否和预料到的相同。\newline
\end{itemize}

\section{真题实战}

\subsection{王道2019年线性表第10题}
若长度为n的非空线性表采用顺序存储结构,在表的第i个位置插入一个数据元素,i的合法值应该是(   )
解:\newline
线性表中的位子序列从{\color{red}1}开始,计算机中的数组从{\color{red}0}开始\newline
隐含条件是原先第i个位子如果有元素,其和其后的元素后移。易知是  $ 1 \le i \le n+1 $

\subsection{2010年408}
设将n(n>1)个整数存放到一维数组R中。试设计一个在时间和空间两方面都尽可能高效的算法。
将R中保存的序列循环左移p(0<p<n)个位置,即将R中的数据由(X0,X1,X2...Xn-1)变换为(Xp,Xp+1...Xn-1,X0,X1,...,Xp-1).要求:\newline
1. 给出算法的基本设计思想.\newline
2. 根据设计思想,采用C或C++语言描述算法,关键之处给出注释。\newline
3. 说明你所设计算法的时间复杂度和空间复杂度. \newline
解:(可能一开始想不出特别好的方法或者当你速度特别慢的时候,记得先把一种最容易的方法写上去,如果是正确的但是时间和空间复杂度不是最优秀的也可以得到10/15分的成绩)\newline
1) 算法设计思想:\newline
1. 先把数组 0 - (p-1) 个反转\newline
2. 再把数组 p - 最后 反转\newline
3. 最后吧整个数组反转\newline
2)\newline
\begin{lstlisting}[basicstyle=\small\ttfamily, caption={}, numbers=none]
#include <iostream>
#include <algorithm>
using namespace std;
// reverse(first, last)
// (1) first表示要排序数组的起始地址;
// (2) last表示数组结束地址的下一位;
// A 数组  p 左移个数  len 数组长度
void shiftLeft(int A[], int p, int len) { // 写出这个函数带上头文件 algorithm 就可以算对了
	reverse(A, A + p);
	reverse(A + p, A + len);
	reverse(A, A + len);
}

int main() {
	int a[] = { 1, 2, 3, 4, 5, 6, 7, 8, 9 };
	shiftLeft(a, 4, 9);
	for (int i = 0; i < 9; i++) {
		cout << a[i] << " ";
	}
	cout << endl;
	system("pause");
}
\end{lstlisting}
3)\newline
时间复杂度\newline
整个序列反转一次,相当于对一半的元素执行了交换,( temp = a; a = b; b =temp;) 执行次数为 n / 2 * 3.总共相当于两次执行整个序列的反转。执行次数是n * 3,时间复杂度易知是 O(n) , 空间复杂度,因为没有额外增加其他的空间,算法原地工作,所以是 O(1).\newline

---------\newline

另解:\newline
1) 算法设计思想:\newline
1. 先把数组 0 - (p-1) 个存放在一个额外的数组中,再原数组向左平移p个单位,最后吧额外数组中的元素复制到后面\newline
2)\newline
\begin{lstlisting}[basicstyle=\small\ttfamily, caption={}, numbers=none]
#include <iostream>
#include <algorithm>
#include <vector>
using namespace std;
// reverse(first, last)
//(1)first表示要排序数组的起始地址;
//(2)last表示数组结束地址的下一位;
// A 数组  p 左移个数  len 数组长度
void shiftLeft1(int A[], int p, int len) {
	reverse(A, A + p);
	reverse(A + p, A + len);
	reverse(A, A + len);
}

void shiftLeft2(int A[], int p, int len) {
	vector<int> v;
	for (int i = 0; i < p; i++) {
		int tmp = A[i];
		v.push_back(tmp);
	}
	for (int i = p,j = 0; i < len; i++, j++) {
		A[j] = A[i];
	}
	for (int i = len - p,j =0; i < len; i++,j++) {
		A[i] = v[j];
	}
}


int main() {
	int a[] = { 1, 2, 3, 4, 5, 6, 7, 8, 9 };
	shiftLeft2(a, 4, 9);
	for (int i = 0; i < 9; i++) {
		cout << a[i] << " ";
	}
	cout << endl;
	system("pause");
}
\end{lstlisting}
3)\newline
时间复杂度\newline
前p个元素复制了两次2*p,后n - p 个元素转移了一次 ,总共 n + p 次,时间复杂度是 O(n)\newline
空间复杂度,增加了p个空间,所以是 O(p) .\newline

\subsection{2009年线性表408}
已知一个带有表头节点的单链表,节点结构为 data, link (杭电考试一般会给出结构体)\newline
假设该链表只给出了头指针list。在不改变链表的前提下,请设计一个尽可能高效的算法,查找链表中倒数第K个位置上的节点(K为正整数)。若查找成功,算法出书该节点的data域的值,并返回1;否则只返回0.\newline
1. 给出算法的基本设计思想.\newline
2. 描述算法的纤细时间步骤.\newline
3. 根据设计思想,采用C或C++语言描述算法,关键之处给出注释。\newline

解:\newline
1)\newline
算法的基本设计思想是:(王道上面写的很详细,照搬了)\newline
关键一遍遍历,设定两个指针,一个指针只管往下遍历,另一个指针,在第一个指针,开始遍历次数统计为k的时候开始遍历。第二个指针如果运动了的话,那么就证明倒数第k个节点存在。\newline
2)\newline
1. 初始化: count = 0, p 和 q 指针指向第一个节点。\newline
2. 开始遍历: p 只要没有遇到 NULL 节点就一直往下遍历,同时count开始计数,当count >= k时q才开始遍历。
3. 判断: 如果count < k时, 说明没有倒数第 k 个节点,返回0结束。否则输出第k个节点的值,然后 返回 1.
3)\newline
\begin{lstlisting}[basicstyle=\small\ttfamily, caption={}, numbers=none]
#include <iostream>
using namespace std;

typedef struct LNode {// 关键处的注释 略
	int data;
	struct LNode *link;
}LNode;

int search_k(LNode* list, int k) {
	LNode *p = list->link, *q = list->link;
	int count = 0;
	while (p != NULL) {
		if (count < k) count++;
		else q = q->link;
		p = p->link;
	}
	if (count < k)
	return 0;
	else {
		printf("%d", q->data);
		return 1;
	}
}

int main() {
	//创建节点
	
	//创建头结点
	LNode *list = (LNode *)malloc(sizeof(LNode));
	list->data = -1;
	list->link = NULL;
	LNode *q = list;
	for (int i = 0; i < 10; i++) {
		LNode *p = (LNode *)malloc(sizeof(LNode));
		p->data = i;
		p->link = NULL;
		q->link = p;
		q = p;
	}
	q = list;
	while (q) {
		cout << q->data << endl;
		q = q->link;
	}
	system("pause");
}
\end{lstlisting}
-------\newline
其他想法:\newline
把遍历的结果存放入一个额外数组中,然后,直接判断 len - k 是否小于 0, 但是结果只能得到 10分。\newline


\subsection{2012年线性表408}
假定采用带头结点的单链表保存单词,当两个单词有相同的后缀是,则可共享相同的后缀存储空间,例如,"loading" 和 "being"的存储映像如下所示。\newline
设str1 和str2 分别指向两个单词所在单链表的头结点,链表的节点结构为[data, next],请设计一个时间上尽可能高效的算法,找出由str1和str2所指向两个链表共同后缀的起始位置。\newline
1. 给出算法的基本设计思想.\newline
2. 根据设计思想,采用C或C++语言描述算法,关键之处给出注释。\newline
3. 说明你所设计算法的时间复杂度. \newline

解:\newline
1)\newline
算法的基本设计思想是:(王道上面写的很详细,照搬了)\newline
1. 计算两个单词的长度。\newline
2. 将 p 和 q 分别指向两个单词,同时将长的单词的指针,和短的指针长度对齐,然后开始遍历直到他们找到共同的节点。\newline
2)\newline
\begin{lstlisting}[basicstyle=\small\ttfamily, caption={}, numbers=none]
typedef struct LNode {
	char data;
	struct LNode *next;
}LNode;
/*求链表长度*/
int listlen(LNode* head) {
	int len = 0;
	while (head->next != NULL) {
		len++;
		head = head->next;
	}
	return len;
}
/* 找出统统后缀的起始地址 */
LNode * find_addr(LNode *str1, LNode *str2) {
	int m, n;
	LNode *q, *p;
	m = listlen(str1); // p 对应 m
	n = listlen(str2); // q 对应 n
	for (p = str1; m > n; m--)
		p = p->next;
	for (q = str2; m < n; n--)
		q = q->next;
	//现在p和q在倒数相同的位置上,确定他们的共同位置
	while (p->next != NULL && p->next != q->next) {
		p = p->next;
		q = q->next;
	}
	return q->next;
}
\end{lstlisting}
3)\newline
时间复杂度:\newline
{\color{red}一般指最坏时间复杂度} 由此得知, 计算两个长度为  m+n, 对齐假设啥都不干, 只有最后一个字符是他们共同的节点,那么就可以得出右遍历了一次, 最大时间复杂度是 $ 2 \times (m + n) $, 所以时间复杂度是 O(m + n). 空间复杂度是O(1), 原地操作

\subsection{王道有意义的题}
3个不同元素一次进栈,能得到( ) 不同的出栈序列。\newline

解:\newline
使用卡特兰数公式:$\frac{1}{3+1}\frac{(6)!}{3! \times 3!} = 5$ 种不同的出栈序列。

\subsection{2014年计算机408}
假设栈初始为空,将中缀表达式a/b+(c*d - e*f)/g 转换为等价的后缀表达式的过程中,当扫描到f时,栈中的元素依次是( ).\newline
解:\newline
栈: push(/),'/' > '+' pop('/') push('+'),push'(', push'*','*' > '-' push '-' pop '*','-' < '*'  push '*', == +,'(',-,'*'\newline
打印序列: a,b,'/',c,d,'*',e,f \newline
继续\newline
栈:pop'*' pop'-' del'(', '+' < '/' push '/',pop '/', pop '+'\newline
打印序列: '*''-',g,'/''+'\newline





