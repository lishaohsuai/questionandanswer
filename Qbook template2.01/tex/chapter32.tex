%# -*- coding: utf-8-unix -*-
%%==================================================
\chapter{mysql}
\label{chap1}
\begin{itemize}[noitemsep,topsep=0pt,parsep=0pt,partopsep=0pt]
	\item ...
\end{itemize}

\section{知识点和方法论}
\subsubsection{索引的使用原则}
对于区分度高的索引建议使用索引.

索引可以提高查询速度

索引降低了数据更新操作的速度

\subsubsection{主键自增}

如果输入了一个值 比如100, 那么下一个从100快开始自增.

\subsubsection{主键自增的好处}
数据库自动编号,速度快,而且是增量增长,按顺序存放,对于检索非常有利;(因为是聚簇索引, 不会造成已经构建好的mysql页进行分裂.) 效率大大提高
\subsubsection{行锁 共享锁}
行锁

select * from XXX  for update;

共享锁

select * from XXX lock in share mode;

\subsubsection{回表}
当你建立了一个索引非主键索引, 那么叶子节点存储的是主键索引所在的地方, 然后会去主键索引中查找真实数据, 这种查询了两个索引就叫回表.
\subsubsection{mybatis\#和\$区别以及原理}
\#{ }可以防止Sql 注入,它会将所有传入的参数作为一个字符串来处理。

\$ {} 则将传入的参数拼接到Sql上去执行,一般用于表名和字段名参数,\$ 所对应的参数应该由服务器端提供,前端可以用参数进行选择,避免 Sql 注入的风险
\subsubsection{redo log 和 undo log, bin log }
如果修改直接修改到日志中的话, 性能开销比较大, mysql 会将要写的数据先写到redolog中,然后再讲redolog存储到磁盘中, 使用场景系统崩溃恢复 \par
但只有 redo log 也不行,因为 redo log 是 InnoDB特有的,且日志上的记录落盘后会被覆盖掉。因此需要 binlog和 redo log二者同时记录,才能保证当数据库发生宕机重启时,数据不会丢失。因为Checkpoint之前的页都已经刷新回磁盘。数据库只需对Checkpoint后的重做日志进行恢复,这样就大大缩短了恢复的时间。

bin log: 归档日志, 通过追加方式记录, 使用场景: 主从复制和数据恢复, 简单理解就是二进制sql数据 \par
undo log: 回滚日志, MVCC多版本并发. 快照读是MVCC的一种体现方式.
比如一条 INSERT 语句,对应一条DELETE 的 undo log.update语句还是update只不过数据是之前的数据,这就是undo log中记录的日志内容。有两个数据, 事物id 和 回滚指针. 如果失败会产生回魂


梳理下事务执行的各个阶段:

(1)写undo日志到log buffer;

(2)执行事务,并写redo日志到log buffer;

(3)如果innodb\_flush\_log\_at\_trx\_commit=1,则将redo日志写到log file,并刷新落盘。

(4)提交事务。



\subsubsection{隔离级别}
\par
读已提交(read committed): 简单来说select * from account, 读出来的数据都是已经提交的数据.
\par 可以出现不可重复读(一个事物两次读到的事物不一样)和幻读(插入一条记录)
\par
读未提交(read uncommitted) : 简单来说 select * from account, 读出来的数据是另一个事物还没有提交的数据,
\par 可以出现脏读(无论是脏写还是脏读,都是因为一个事务去更新或者查询了另外一个还没提交的事务更新过的数据。因为另外一个事务还没提交,所以它随时可能会回滚,那么必然导致你更新的数据就没了,或者你之前查询到的数据就没了,这就是脏写和脏读两种场景。): 简单的说, 就是A事物在还没commit数据的情况下, b事物使用了A事物的数据, 但是A事物出现了回滚操作, 导致B事物读出来的数据是脏数据. 可以出现不可重复度, 可以出现幻读. (感觉是最不靠谱的一个级别)
\par
可重复读(RR: Repeatable read) : 简单来说, 就是A事物插入了一条新数据, B事物没看到, 因为可重复读, 但是B是可以操作这条新数据的.
\par
串行化(serializable): 只能一个一个处理, 没什么问题, 但是效率太低一般不考虑.
\par 提示: 不可重复读对应的是修改即Update,幻读对应的是插入即Insert.
\par TIPS:
可见,幻读就是没有读到的记录,以为不存在,但其实是可以更新成功的,并且,更新成功后,再次读取,就出现了。
\subsubsection{ACID}
原子性: 不可分割
\par
一致性: 数据完整性, 例如金钱的综述
\par
隔离性: 不被打扰
\par
持久性: 永久保存
\subsubsection{乐观锁和悲观锁}
悲观锁, 指的是在整个数据处理过程中, 将数据处于锁定状态. 悲观锁的实现, 往往依靠数据库提供的锁机制.
\par
乐观锁的实现, 基于并发控制的CAS理论.
\subsubsection{MVCC}
MVCC用于提交读和可重复读这两种隔离级别. 使用undolog实现
\par
主要依靠事务版本号来实现读已提交和可重复读.
\par
当开始一个新事物时, 该事物的版本号肯定会大于当前所有数据行快照的创建的版本号
\subsubsection{RR和RC隔离级别下的InnoDB快照读有什么区别}
(1) RR隔离级别下, 当事物第一次进行快照读, 仅此一次创建read-view视图, 所以read-view
中未提交事物数组和最大事物ID始终保持不变, 因此每次读取时只会读取事物之前的数据 \par
(2) 而在RC隔离级别下, 每次事物进行快照读是, 都会生成新的read-view视图, 导致在RC隔离级别下事物可以看到其他事物修改后的数据, 这也是导致不可重复的原因 \par
总之, 在RC隔离级别下, 是每个快照读都会生成并获取最新的Read-View: 而在RR隔离级别下, 则是同一个事物中的第一个快照读才会创建Read View, 之后的快照读获取的都是同一个Read View \par
\subsubsection{B+/B树之间的比较}
(1) B+树空间利用率更高, 可减少IO次数. \par
(2) B+树所有关键字的查询路径长度相同(因为关键字在叶子节点), 导致每一次查询的效率相当, 更加稳定. \par
(3) B+叶子节点有指针, 支持between查找很方便.
\subsubsection{聚集索引\&非聚集索引}
简单来说, 一个表只有一个聚集索引, 类似于主键索引.

一个表可以有多个非聚集索引, 打比方, 新华字典A-Z的查询事主键索引, 新华字典偏旁查询事非聚集索引. 是会跳跃的.


\subsubsection{创建索引的优点}
(1) 当数据量增大的时候, 索引可以极大的加快数据的查找速度
\subsubsection{创建索引的缺点}
(1) 对于一个表而言, 不单单要维护数据的存储, 也要维护索引的存储. (空间增大)
\par
(2) 对表中的数据进行修改的时候, 索引也要进行动态维护, 这样就降低了数据的维护速度.
\subsubsection{MYSQL优化}
(1) 最左前缀法则 \par
(2) 较长的数据列建立前缀索引 \par
(3) 常查询数字建立索引或者组合索引 \par
(4) 分解大连接查询, 分解成对每一个表进行一次单表查询. 可以对缓存更好效的应用. 即时其中一个表发生变化, 对其他表的查询缓存依然可以使用.
\subsubsection{InnoDB \& MyISAM}
(1) myISAM 支持全文索引, innodb 也支持全文索引, 简单来说innodb 比myISAM强太多了\par
另外inndb 对于全文索引有要求, 有一个最小搜索长度. 和最大搜索长度. 比like 之类快很多.
\subsubsection{创建存储过程}
(1) 创建存储过程比单独的sql语句要快\par
(2) 可以快速进行测试
\subsubsection{热备份和冷备份}
冷备份: 因为mysql是基于文件的, 所以, 我们在mysql关闭的时候, 直接使用 copy 拷贝一份即可. \par
热备份: 使用mysqldump 命令对 正在运行的mysql程序的数据库进行备份
\subsubsection{Innode加锁}
(1) 对于update,delete和insert语句, Innorb会自动给设计数据集加排它锁(X), 对于普通select语句Innodb不会加任何锁; 事物可以通过以下语句显示给记录集加共享锁或排他锁.
(2) 共享锁(S): select * from tableName where ... LOCK IN SHARE MODE(可以查看但无法修改和删除一种数据锁, 其他用户可以加共享锁) \par
(3) 排它锁(X): SELECT * from tableName where ... FOR UPDATE(独占锁, 其他任何事物都不能对A加任何类型的锁, 直到自己释放了锁.) \par
共享锁, 主要用在确保没人对这个记录进行UPDATE或者DELETE操作.
\subsubsection{INNODB解决死锁}
实际上在INNODB发现死锁之后, 会计算出两个事物各自插入,更新或者删除的数据量来判定两个事物的大小. 也就是哪个事物所改变的记录条数越多, 在死锁中就越不会被回滚掉.
\subsubsection{Mysql锁你了解哪些}
按照锁的粒度区分 \par
1. 行锁, 锁某行数据, 锁力度最小, 并发度最高 \par
2. 表锁, 锁整张表, 锁力度最大, 并发度低 \par
3. 间隙锁, 锁的是一个区间 \par
按照怕他性区分 \par
1. 共享锁: 也就是读锁 \par
2. 排它锁: 也就是写锁 \par
还可以分为: \par
1. 乐观锁, 并不会真正的去锁某行记录, 而是通过一个版本号来实现的. \par
2. 悲观锁, 上面所说的行锁, 表锁, 都是悲观锁. \par
\subsubsection{Mysql数据库中, 什么情况下设置了索引但是无法使用?}
1. 没有符合最左前缀原则 \par
2. 字段进行了隐私数据类型转化 \par
3. 走索引没有全表扫描效率高