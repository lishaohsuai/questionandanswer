%# -*- coding: utf-8-unix -*-
%%==================================================
\chapter{自我}
\label{chap1}
\begin{itemize}[noitemsep,topsep=0pt,parsep=0pt,partopsep=0pt]
	\item ...
\end{itemize}

\section{知识点和方法论}
\subsubsection{自我介绍}
面试官好, 我是杭州电子科技大学, 计算机科学与技术专业的学生, 来自浙江温州,
研究生主要研究的方向是六面体网格生成. 发表了一篇中文核心的论文.
研究生期间参加过多种项目, 典型的是参加过之江实验室的天枢深度学习可视化的项目, 作为前端开发人员。
同时也积极参与多种竞赛. 参加过华为软件精英挑战赛和数学建模比赛. 数学建模是国二. 华为软件精英挑战赛今年拿过杭夏赛区的前64强. \par
我的兴趣爱好是跑步,参加过西湖毅行.
\subsubsection{自我介绍英文版本}
Hello everyone. I am liyongjie, a computer science and technology student from HangZhou Dianzi university. I grew up in WenZhou, but now I'm living in Hangzhou. I participated in several games including the Huawei Software elite Challenge, and mathematical modeling. More importantly I learnt how to communicate well with my team members, and how to division of labor Challenge. Then I participtated in several projects, including zhijiang Deep learning visualization. I worked one year and I was a software development. I like running in my free time and I am so happy to be here and looking forward to working with you in the near future.

\subsubsection{优缺点}
优点是: 坚持且不会轻易放弃, 只要认为能做完可以做完, 我会坚持做完它, 比如数学建模, 我们做的是比较慢, 最后一天12点我们队伍还在做第四问, 队友劝我放弃吧, 先把前三问完成. 我那个时候有思路了, 然后我说等我2个小时, 我一定可以做完, 结果花了30多分钟就做好了, 不过调试花了挺久的时间

缺点是: 有一定拖延症, 有的时候等到任务紧要的时候会效率比较高.

\subsubsection{如何看待华为文化}
狼性文化, 胜则举杯相庆 败则拼死相救; 告诉每一个华为人, 要追逐胜利不惧失败

\subsubsection{遇到的困难以及解决办法?}
首先遇到问题, 根据问题去查找有没有相关的解决方案.

其次, 如果没有解决方案, 应该有相关知识点. 根据相关知识点去搜相关论文.

如果还没有解决方案, 可以和朋友和老板沟通交流一下. 有的时候在沟通交流的时候就会出现解决方案.

\subsubsection{未来3~5年的规划和职业目标是什么?}

前两年, 好好学习掌握华为的技术, 后面几年争取输出技术.

职业目标是: 能够独立的带领团队开发项目.

\subsubsection{加入不同项目组认为自己的板块是最重要的,没有及时给你配合或者你遇到配合上,进度上的问题如何解决?}

一方面对自己上下游的工作内容要有一定的熟悉, 如果是上游出现了问题, 那只能把问题反馈给上游, 术业有专攻.

二是, 在别人有空的时候要经常去找对方沟通交流, 如果是要配合的一些小事, 如果能自己干的话就自己干了.

\subsubsection{反问}

部门的发展方向是什么?

什么时候有通知


\subsubsection{为什么读研}
1. 一方面想要完成自己的小小愿望, 在大学期间拿一个奖项. 本科也参加过电子设计竞赛, 但是这种比赛比较吃实验室的资源. 学计算机的话, 比如数学建模, 只要一台电脑就可以了.

2. 工资比较低, 感觉不如读研, 然后找个好工作.

\subsubsection{用英语简单聊}
