%# -*- coding: utf-8-unix -*-
%%==================================================
\chapter{系统设计题}
\label{chap1}
\begin{itemize}[noitemsep,topsep=0pt,parsep=0pt,partopsep=0pt]
	\item ...
\end{itemize}

\section{知识点和方法论}
\subsubsection{分布式ID生成器}

使用UUID
4个字节表示的Unix timestamp,
3个字节表示的机器的ID
2个字节表示的进程ID
3个字节表示的计数器

SnowFlake 算法,是 Twitter 开源的分布式 id 生成算法。其核心思想就是:使用一个 64 bit 的 long 型的数字作为全局唯一 id。在分布式系统中的应用十分广泛,且ID 引入了时间戳,基本上保持自增的,后面的代码中有详细的注解。

给大家举个例子吧,比如下面那个 64 bit 的 long 型数字:

第一个部分,是 1 个 bit:0,这个是无意义的。

第二个部分是 41 个 bit:表示的是时间戳。

第三个部分是 5 个 bit:表示的是机房 id,10001。

第四个部分是 5 个 bit:表示的是机器 id,1 1001。

第五个部分是 12 个 bit:表示的序号,就是某个机房某台机器上这一毫秒内同时生成的 id 的序号,0000 00000000。