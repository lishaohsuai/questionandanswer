%# -*- coding: utf-8-unix -*-
%%==================================================
\chapter{数量关系总览}
\label{chap1}
\begin{itemize}[noitemsep,topsep=0pt,parsep=0pt,partopsep=0pt]
	\item ...
\end{itemize}

\section{知识点和方法论}
\subsection{分数}
$$
	\frac{1}{2} = 0.5
$$
$$
	\frac{1}{3} = 0.33
$$
$$
	\frac{1}{4} = 0.25
$$
$$
	\frac{1}{5} = 0.2
$$
$$
	\frac{1}{6} = 0.167
$$
$$
	\frac{1}{7} = 0.142
$$
$$
	\frac{1}{8} = 0.125
$$
$$
	\frac{1}{9} = 0.111
$$

$$
	\frac{1}{125} = 0.008
$$

\subsection{圆}
$tan15 = 0.268$
\subsection{空瓶换酒}
M-1个空瓶可以换酒
\subsection{路程}

$$
	60km/h = 60 / 3.6 =
$$
\subsection{平年, 闰年}
平年星期 + 2

闰年 星期 + 1
\subsection{倍数}
$$
	2^2 = 4;
	3^2 = 9;
	4^2 = 16;
	5^2 = 25;
	6^2 = 36;
	7^2 = 49;
	8^2 = 64;
	9^2 = 81;
	11^2 = 121;
	12^2 = 144;
	13^2 = 169;
	14^2 = 196;
	15^2 = 225;
	16^2 = 256;
	17^2 = 289;
	18^2 = 324;
	19^2 = 361;
$$



\subsection{奇偶特性}
简单来说就是从奇偶特性推导出答案

1. 奇数 $\pm$ 奇数 = 偶数; 偶数 $\pm$ 偶数 = 偶数;

偶数 $\pm$ 奇数 = 奇数; 奇数 $\pm$ 偶数 = 奇数;

2. 奇数 $\times$ 奇数 = 奇数; 奇数 $\times$ 偶数 = 偶数;
偶数 $\times$ 奇数 = 偶数; 偶数 $\times$ 偶数 = 偶数;


\subsection{倍数特性}
1. 3, 9 整除判定基本法则

一个数能被3整除, 当且仅当其各位数字之和能被3整除;

一个数能被9整除, 当且仅当其各位数字之后能被9整除;

2. 2, 4, 8, 5, 25, 125 整除判定基本法则

一个数能被2(或者5)整除, 当且仅当末一位数字能被2(或者5)整除;

一个数能被4(或者25)整除, 当且仅当末两位数字能被4(或者25)整除;

一个数能被8(或者125)整除, 当且仅当末三位数字能被8(或者125)整除;

\subsection{方程法}

设未知数时候, 一般采取设小不设大

加速


$$
	\frac{a}{c} = \frac{b}{d} = \frac{a+b}{c+d} = \frac{a-b}{c-d}
$$


\subsection{赋值法}

一般对效率, 成本, 进价等赋值时, 昌吉和比例关系赋值简单数, 数字要尽可能地便于计算和化简
, 如1, 2, 60, 100 等.

赋0法, x+y+z 一定, 那么直接设其中一个值为0

\subsection{等距离平均速度公式}

$$
	\mbox{等距离平均速度} = \frac{2v_1 v_2}{v_1 + v_2}
$$

\subsection{利润率问题}

$$
	\mbox{利润率} = \mbox{利润} \div \mbox{进价}
$$

\subsection{溶质问题}
$$
	\mbox{浓度}=\frac{\mbox{溶质质量}}{\mbox{溶液质量}}
$$

\subsection{排列组合}
$$
	A^{3}_{10} = 10 \times 9 \times 8
$$

$$
	C^{3}_{10} = \frac{10 \times 9 \times 8}{3 \times 2 \times 1}
$$

\subsection{几何}

球的表面积 = $4\pi R^2$

球的体积 = $\frac{4}{3} \pi R^3$

\subsection{周期问题}
每n天的一个周期, 为n天

每隔n天的一个周期是(n+1)天




\subsection{全错误排列}

主体为4辆车, 要求所有车都不得停在原来的车位中, 则一共有多少种不同的停放方式.

记住结论 $D_1 = 0, D_2 = 1, D_3 = 2, D_4 = 9$


\subsection{三集合容斥原理}

$A + B + C - A\bigcap B - A \bigcap C - B \bigcap C + A \bigcap B \bigcap C = \mbox{总数} - \mbox{A, B, C, 均不满足的个数} $

$$A + B + C = a + 2b + 3C$$

a,b,c分别表示满足一个条件的数量和满足2个条件的数量和满足三个条件的数量

$$
	A + B + C - b - 2c = \mbox{总数} - \mbox{A, B, C均不满足的个数}
$$


\subsection{追及问题}

300m环形跑道上从同一点触发 同向而行, 每追上一次则多跑一圈. A 每次追上B后都减速0.5m/s, 当A从6m/s减速到3m/s,
根据
$$
	S_{追} = 300 = (V_A - V_B) \times T
$$

共同跑的路程是990m

\subsection{代入排除法}
1. 尾数法加速排除

2. 代入排除不是无脑代入, 需要有所选择. 一般题目问最大, 那我们就考虑从最大的开始代入, 问
最小就从最小 开始代入.

\section{数量关系补充}
\subsection{蒙题大法}
1. 问最大, 蒙次大

2. 问最小, 蒙次小或最大

3. 3+1蒙题:三项等差/等比数+1项特殊项,蒙接近特殊项的一项

4. 迷惑项蒙题, 看问题, 和差倍比关系

5. 找共性蒙题

6. 常识蒙题

\subsection{例题: 圆桌公式}
2个大人带4个小孩去坐旋转木马, 问不相邻的概率

$$
	A^{n-1}_{n-1}
$$

$$
	A^{3}_{3} \times A^{4}_{2} \div A^{5}_{5} = 72 \div 120
$$

\subsection{例题: 周期公式}
网管小刘负责A,B,C三个机房的巡查工作, A,B,C分别需要每隔2,4,7天巡检一次, 3月1日, 小刘巡检了三个机房, 问他在整个3月有几天不用做机房的巡检工作

TIPS: 每隔N天=每(N+1)天

在3月份身上下的30天中, A需要  $\frac{30}{3} = 10$, B需要 $\frac{30}{5} = 6$, C $\frac{30}{8} = 3$

A 和 B $\frac{30}{15} = 2$

B 和 C $\frac{30}{40} = 0$

A 和 C $\frac{30}{24} = 1$

A 和 B 和 C 0
10 + 6 + 3 -2 - 1 = 16天, 休息 14天


\subsection{例题: 单端触发相遇问题}

$$
	2nS = (v_1 + v_2) t
$$

n 相遇次数, t 时间

小王和小李沿着绿岛往返运动, 绿道总长度为3km. 小王没小时走2km; 小李每小时跑4km. 如果两人同时从绿道的一端触发, 则当两人第7次相遇时, 距离触发点(多少公里)
