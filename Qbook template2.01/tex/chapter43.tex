%# -*- coding: utf-8-unix -*-
%%==================================================
\chapter{判断推理总览}
\label{chap1}
\begin{itemize}[noitemsep,topsep=0pt,parsep=0pt,partopsep=0pt]
	\item ...
\end{itemize}

\section{知识点和方法论}
\subsection{推理}

1. 如果    要

如果\textbf{去新疆}, 就要\textbf{游吐鲁番和喀纳斯}

=> 去新疆 -> 游吐鲁番和喀纳斯

2. 只有   才 (后推前)

只有\textbf{与小李同游}, 小张才\textbf{会游吐鲁番或天池}

=> 游吐鲁番或天池 -> 与小李同游

3. 否后比否前


游吐鲁番或天池 -> 与小李同游 -> 与小李做约定 -> 小李这个夏天一定有时间

否后比否前, 推出 -(游吐鲁番或天池), 根据德摩根定律, -吐鲁番且-天池. 推出小张没有去新疆

4. 德摩根定律

-(A或B) = -A 且 -B


5. 或 或

否一, 推一

只有拥有强大的科技创新能力, 才能增强综合国力, 才能提高国际竞争力. = 国际竞争力 -> 强大的科技创新能力

我国或者国际竞争力得不到提高, 或者拥有强的科技创新能力. = -国际竞争力 或 有强的科技创新能力 => 国际竞争力 -> 强大的科技创新能力

5. 既是 ... 更是 ...

贫困群众既是脱贫攻坚的扶贫对象更是脱贫致富的主体力量.

贫困群众 -> 脱贫攻坚的扶贫对象 且 脱贫致富的主体力量

6. 所有... 都

前对后

所有脱贫攻坚的扶贫对象都是脱贫致富的主体力量.

\subsection{图形推理}

1. 思路

a. 各图形构成相同, 一般考察位置规律

b. 各图形构成相似, 一般考察样式规律

c. 各图形构成不同, 一般考察属性, 数量及其他特殊规律

d. 圆相切和相交, 可以考虑一笔画

\subsection{九宫格图推}
1. 如果平移元素只在九宫格或十六宫格图形的最外圈出现, 有限考虑元素在最外圈按时针方向平移.

2. 如果平移元素出现在非最外圈位置, 优先考虑直线方向平移.

\subsection{翻转}

\subsection{样式规律}
题型特征: 图形元素组成相似

1. 加减同异

相加:将两图形中所有元素拼合成一幅新图形

相减: 当第一幅图的元素或线条完全包含第二幅图是, 两图相减的结果, 就是第一幅图去掉第二幅图所有元素之后的图形

求同: 将两图形中所有不同的元素去掉, 只留下相同的部分, 形成一幅新图形.

求异: 将两图形中所有相同元素去掉, 只留下各自不同的部分形成一幅新图形.


2. 缺啥补啥

当各图形组成相似, 且某些元素或特诊不止一次出现在各图形中时, 优先考虑缺啥补啥.


3. 叠加运算

一般要列出四个公式

白 + 白 = ?

黑 + 黑 = ?

黑 + 白 =

白 + 黑 =

\subsection{属性规律}
题型特征: 图形元素组成不同

1. 对称性

a. 轴对称

b. 中心对称

c. 既是轴对称也是中心对称

d. 对称轴的方向和数量

2. 曲直性

a. 曲: 图形只由曲线构成

b. 直: 图形只由直线构成

c. 曲+直:图形由曲线和直线共同构成

3. 开闭性

a. 开放图形, 不行不包含任何封闭空间, 即没有窟窿

b. 封闭图形, 图形包含封闭空间, 即有窟窿


\subsection{数量规律}
题型特征: 图形元素组成不同, 且无明显属性规律.

1. 点

切点, 直线和曲线的交点.

2. 线

当图形中出现多边形或单独的一条直线时, 优先考虑数直线; 当图形中出现较多曲线时, 优先考虑数曲线.

3. 数笔画

a. 连通图的笔画数 = 奇点数 / 2;

奇点: 若以一个点为起点, 延伸出的线条数为奇数, 则该店为奇点

b. 长考笔画规律的特征图形: 五角星, 月亮, "日", "田", 当看到题目中出现以上特征图形时, 可优先考虑数笔画.

4. 数面

5. 数元素

元素的个数, 种类, 部分数

6. 数角
\subsection{特殊规律}

图形间的规律: 相离, 相压

相交

a. 相交于面, 相交于点, 相交于边

\subsection{空间重构}
1. 六面体

在立体图智能看到一个对面, 不能看到两个对面

通过在公共边顺时针画一圈

三个面相邻公共点不变

2. 截面图


3. 三视图

\subsection{定义判断}
1. 关键词

主客体

关联词

方式+目的端口

a. 按照 / 通过 / 采用 / 利用 的方式 方法 / 办法/ 依据 / 手段

b. 以 / 达到 / 实现

2. 同构选项进行排除

\subsection{类比推理}
1. 语义关系

a. 近义词

b. 反义词

c.  比喻词

2. 全同关系

一年四季, 春夏秋冬.

3. 包容关系

a. 种属关系

b. 组成关系

4. 并列关系

a. 容斥关系

b. 非容斥关系

5. 交叉关系

6. 对应关系

a. 配套使用, 牙刷 \& 牙膏

b. 物品与原材料, 制作工艺.

c. 物品与功能

d. 属性关系

e. 因果关系

7. 语法关系

a. 词性: 名词, 动词, 形容词

b. 顺序: 题干和选项用同样的顺序造句.

\subsection{翻译推理}

\subsubsection{前推后}
1. 如果 ... 就 ...

2. 只要 ... 就 ...

3. 所有 ... 都 ...

4. ... 是 ... 的充分条件

5. ... 就/则/都/一定...

\subsubsection{后推前}

1. 只有 ... 才..

2. 不 ... 不 ...

3. ... 才 \dots

4. 除非 ... 否则不 \dots

要么不回答, 除非含糊不清

回答 -> 含糊不清

除非田宇去海洋社区, 否则王栋不去文明社区

王栋去文明社区 -> 田宇去海洋社区

5. ... 是 ... 的必要条件

\subsubsection{推理规则-逆否等价}

a -> b

=>

-b -> -a

\subsubsection{传递规则}

\subsubsection{tricks}

谁必不可少, 谁在箭头后面

德摩根定律

-(a 且 b) = -a或-b

\subsection{排列组合}

代入排除不是无脑代入


最大信息法

列表

\subsection{加强与削弱}

\subsubsection{削弱}
否定论点

拆桥

论点: 有1就有2

削弱方式1 -- 否定论点: 有1同时没有2

削弱方式2 -- 可能性拆桥: 没有1同时有2

方式1比方式2削弱力度强

削弱方式3 否定论据

削弱方式4  因果倒置与另有他因

1. 支持方: 1是2的原因; 反对方: 1不是而的原因, 三是二的原因

2. 削弱反对方: 1是3的原因, 故而1也是2的原因.

否论点强于否论据


\subsubsection{加强}

加强方式1  搭桥

加强方式2 必要条件

加强方式3 解释与举例

\subsubsection{日常结论}
1. 三不选

a. 存在逻辑错误的选项一定不选

b. 无中生有的选项一定不选

c. 偷换概念的选项一定不选

2. 两慎选

a. 概念扩大

b. 有敏感词的慎选, 绝对

3. 一定选

可能

\section{判断推理}

\subsection{例题}
如果\textbf{去新疆}, 就要\textbf{游吐鲁番和喀纳斯}

=> 去新疆 -> 游吐鲁番和喀纳斯

只有\textbf{与小李同游}, 小张才\textbf{会游吐鲁番或天池}

=> 游吐鲁番或天池 -> 与小李同游

如果\textbf{与小李同游}, 小张一定要\textbf{与小李做约定}

=> 与小李同游 -> 与小李做约定

如果\textbf{小张与小李做约定}, 则\textbf{小李这个夏天一定有时间}

=> 小张与小李做约定 -> 小李这个夏天一定有时间

遗憾的是, 小李没去

总路线

游吐鲁番或天池 -> 与小李同游 -> 与小李做约定 -> 小李这个夏天一定有时间

否后比否前, 推出 -(游吐鲁番或天池), 根据德摩根定律, -吐鲁番且-天池. 推出小张没有去新疆

\subsection{正方形}

1. 对面

2. 顺时针画圈
