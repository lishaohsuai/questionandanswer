%# -*- coding: utf-8-unix -*-
%%==================================================
\chapter{资料分析总览}
\label{chap1}
\begin{itemize}[noitemsep,topsep=0pt,parsep=0pt,partopsep=0pt]
	\item ...
\end{itemize}

\section{知识点和方法论}

\subsection{速算技巧}

分子和分母小
$$
	\frac{1+8.6\%}{1+12.5\%} = 1- 4\%
$$

特殊分数

$$
	\frac{\mbox{现期量}}{1+\frac{1}{8}} * \frac{1}{8} = \frac{\mbox{现期量}}{9}
$$

$$
	\frac{\mbox{现期量}}{1-\frac{1}{8}} * \frac{1}{8} = \frac{\mbox{现期量}}{7}
$$

\subsection{增长量比较}

Ar1 ?= Br2



\subsection{术语}
同比, 是今天6月对比去年6月.

环比, 是今天6月对比去年5月.

\subsection{特定历史时期}

九五计划 1996 - 2000年

十三五计划 2016 - 2020年

\subsection{结构阅读法}

略读资料, 详读结构.

\subsection{间隔增长率}

2013年相对于2011年增长了

$$
	r_{间隔增长率} = r_1 + r_2 + r_1 \times r_2
$$

\subsection{年均增长率}

$$
	\mbox{基期量} \times (1+\mbox{年均增长率})^n = \mbox{现期量}
$$


\subsection{基期比重}
部分/整个

2018年 A , B , a, b

2017年 A/(1+a) , B /(1+b)

2017年比重 A/B*((1+b)/(1+a))

\subsection{平均数}

平均每(分母)


\subsection{平均值增长率}
2017 A a

2017 B b

2018 A(1+a)

2018 B(1+b)

$$
	r = (\frac{A(1+a)}{B(1+b)} - \frac{A}{B}) \div \frac{A}{B} = \frac{1+a}{1+b} - 1 = \frac{a - b}{1+b}
$$

\subsection{年均增长量}

$$
	\mbox{年均增长量} = \frac{\mbox{现期量(2020)-基期量(2011)}}{\mbox{年份差(2020-2011)}}
$$


\subsection{多多少倍数}

n倍 - 1 个


\subsection{基期倍数公式}

$$
	基期倍数 = \frac{A}{1+a} \div \frac{B}{1+b} = \frac{A}{B} \times \frac{1+b}{1+a}
$$

A B 表示 现期量

a,b 表示对应增长率

\subsection{两期增长比重}

$$
	\mbox{两期比重差} = \frac{A}{B} - \frac{A}{B} \times \frac{1+b}{1+a} =
	\frac{A}{B} \times \frac{a-b}{1+a}
$$
2018  , A, B, a, b
2017, A/(1+a), B/(1+b)

A 部分, B 整理, 一般选择小的


\subsection{平均增长量}

$$
	\mbox{平均数增长量} = \mbox{现期平均数} - \mbox{基期平均数} = \frac{A}{B} - \frac{A}{B} \times \frac{1+b}{1+a} =
	\frac{A}{B} \times \frac{a-b}{1+a}
$$

其中A表示现期总量, B表示现期个数.

\subsection{两期比重差}
$$
	\mbox{两期比重差} = \frac{A}{B} \times \frac{a - b}{1+a}
$$

\subsection{平均增长率}

$$
	\mbox{平均数增长率} = \frac{\mbox{现期平均数} - \mbox{基期平均数}}{\mbox{基期平均数}} =
	\frac{\frac{A}{B}-\frac{A}{B} \times \frac{1+b}{1+a}}{\frac{A}{B} \times \frac{1+b}{1+a}} = \frac{a-b}{1+b}
$$

其中A表示现期总量, B表示现期个数.

\subsection{加速计算}
1. 截位法, 简单来说就是被除数, 进行保留2-3位进行计算加速.

看选项差距, 选项差距大, 保留2位, 选项差距小保留三位.

2. 估算公式

$$
	\frac{A}{1\pm r} \doteq A(1 \mp r)
$$

当 r < $5\%$

