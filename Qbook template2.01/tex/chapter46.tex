%# -*- coding: utf-8-unix -*-
%%==================================================
\chapter{言语总览}
\label{chap1}
\begin{itemize}[noitemsep,topsep=0pt,parsep=0pt,partopsep=0pt]
	\item ...
\end{itemize}

\section{知识点和方法论}


\subsection{中心理解题}

注意:

关联词, 主题词, 程度词.

转折之后的新观点为本文的重点.

因果关系之后的结论是重点

必要条件, 只有...才 之间的中点

解决问题的对策是重点

反面论证, 简而言之, 就是对策

并列关系, 答案中要都有

中心句, 常见于段首和段尾


\subsection{细节题的容易错误的点}
1. 无中生有

2. 偷换概念

3. 偷换时态

4. 程度轻重

\subsection{排序}

\subsubsection{确定首句}
1. 定义排第一句

2. 背景引入

3. 提出观点

\subsubsection{确定尾句}

引入结论和提出对策的语句通常适合作为尾句

典型标志词: 因此, 所以, 看来, 这+应该, 需要

\subsection{篇章阅读题}
1. 明确阅读顺序, 先根据题干判断题型优先级

2. 利用题干关键词定位原文

3. 合理调整做题顺序, 不一定位, 细节理解题可以考虑放弃

4. 依托重点提示把握文段结构

5. 把握做题时间

\subsection{定义}

妄言: 胡言乱语, 瞎编

推脱: 推卸责任

推托: 婉言拒绝

负担: 一般指消极的事物

\section{申论总览}

\subsection{规范词梳理}

1. 发展特色产业

2. 提高竞争力

3. 提高抗风险能力

4. 稳定就业

5. 按需设岗

6. 改善办学条件

\subsection{tricks}

不懂题干在说什么摘抄对策就完美了

好的影响(例子) + 坏的影响(例子) + 观点(主题)

举例子

化大为小, 将很抽象的概念用很多的小概念进行概括

题目: 动词 + 主题词


\section{言语}

\subsection{tricks}

用词语的感情色彩来判断.

\subsection{成语积累 \& 词语积累}
方兴未艾: 多形容新生事物正在蓬勃发展.

独善其身: 指不做官, 就搞好自身的修养. 现在也指只顾自己, 缺乏集体精神.

以邻为壑: 指那邻国当做大水坑, 把本国的洪水排泄到那里去, 比如吧困难或灾祸退给别人.

手足无措: 形容举动慌张, 侧重于强调处于非常窘迫的外在状态.

亦庄亦谐: 指讲话或文章既庄重正派, 又幽默活泼.

面面俱到: 指各方面都要照顾到, 没有遗漏

等量齐观: 有差别的事物同等看待

缘木求鱼: 比喻方法不对头, 不可能达到目的.

羁绊: 起到消极作用.

羔羊跪乳: 比如乌鸦反哺, 孝敬长辈

言之凿凿: 说明有证据

彩衣娱亲: 比喻孝顺.

墨菲定理: 如果事情有变坏的可能, 不管这个可能性有多小, 它总会发生.

苦心孤诣: 指莎菲苦心地钻研, 到了别人所达不到的至高境界.

青灯黄卷: 形容清苦的攻读生活.

一往而深: 对人或事物倾注了恒盛的感情, 向往而不能克制

久久为功: 持之以恒, 锲而不舍, 驰而不息

斑驳陆离: 形容颜色杂乱的样子

坐而论道: 口头说说, 不见行动, 侧重空谈

老调重弹: 说过多次的理论, 主张重新搬出来, 不能体现出"没有接受教训"的含义, 与文意不符, 排除.

自成一体: 在书法, 绘画等方面具有独创风格, 能自成体系.