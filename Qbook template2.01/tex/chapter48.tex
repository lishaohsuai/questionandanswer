%# -*- coding: utf-8-unix -*-
%%==================================================
\chapter{申论}
\label{chap1}
\begin{itemize}[noitemsep,topsep=0pt,parsep=0pt,partopsep=0pt]
	\item ...
\end{itemize}

\section{知识点和方法论}

\subsection{xi主书记名言}

绿水青山就是金山银山


\subsection{2022年国考}
时间长河奔腾不息,当今中国,正处在由大向强的“关键一跃”,夺目的成就令人骄傲。但是在这继往开来的新时代,仍然面临许多挑战,国际封锁打压仍在持续,国内乡村振兴方才起步、经济下行压力渐增,如何破局?我认为要做好融合的大文章。

共融造就共荣。人类文明演进万年,每一次生产要素的流动与融合都极大地推进了时代的进化。火与食物的融合,开发了人类的大脑,使智慧迸发﹔铁与土地的融合,促进了粮食的生产,使人口增加﹔煤炭电力与机器的融合,推动了创新,使科技爆炸。融合造就了繁荣的历史,我们应毫无疑问地充分肯定融合的价值,用好融合的智慧开辟更为繁荣的未来。正是在这一智慧的指引下,中国处处正在发生着融合更新。从地区上看,大湾区融合激发了粤港澳三地的活力;从地域上看,城乡互动造就了同频共振的欣欣向荣;从产业上看,文旅与交通的融合缔造出无数的网红景点。


融合需要融和。融合不是相加而是相融,不是简单的拼凑嫁接,而是渗透交织、融化和谐。倘若只是将要素简单拼凑,并不会产生良性化学反应,反而可能使各要素相互攻击、彼此掣肘,可谓适得其反、得不偿失。融和的关键在于统筹,统筹意味着分工,能使得各要素各就其位、各展所长,比如“大蜀道文化联盟”的成立吸引了30余个市县区,通过共同开发相互配合,通过各展所长错位竞争,实现了共荣共赢;统筹也意味着部署,能保证各要素各美其美、相互浸润,比如F市第三中学大胆创新探索,构建“德智体美劳”五大类课程群,建立课程有机融合系统,培养出一批批德智体美劳全面发展的优质青年。

融合更需要融活。融合不是目的,融活才是目的,任何要素的交融理应产生更有活力的结果,惟有如此,融合才是有价值的。如何才能保证融活?我认为需要建立机制。通过机制订立规矩、搭建渠道,使各要素有序流动、按需交融,好比人之血管维系人之生机。山南乡农村公路养护因缺少运作机制,政府部门虽然共管一事但各自为政,养护效率极为低下,养护结果问题百出。与之相反,Z城不靠海、不沿边、无矿产、缺交通,原生的发展要素极度匮乏,却能依靠当地政府搭建起的城乡要素融合机制,引导工商资本下乡“输血”,引进工业资本反哺农业,打造出三产交贯融合全产业链体系,让农民增加收入,让土地焕发活力。两相对比答案自明,必须建立要素流动机制保障融活。


回望历史, 时代大吵由浪花卷懂; 望眼前路, 伟业大道以石子路铺就. 让我们始终站在时代潮头, 以融合与融活的只会绘就新时代的精彩篇章.